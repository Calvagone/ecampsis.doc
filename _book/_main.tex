% Options for packages loaded elsewhere
\PassOptionsToPackage{unicode}{hyperref}
\PassOptionsToPackage{hyphens}{url}
%
\documentclass[
]{book}
\usepackage{amsmath,amssymb}
\usepackage{iftex}
\ifPDFTeX
  \usepackage[T1]{fontenc}
  \usepackage[utf8]{inputenc}
  \usepackage{textcomp} % provide euro and other symbols
\else % if luatex or xetex
  \usepackage{unicode-math} % this also loads fontspec
  \defaultfontfeatures{Scale=MatchLowercase}
  \defaultfontfeatures[\rmfamily]{Ligatures=TeX,Scale=1}
\fi
\usepackage{lmodern}
\ifPDFTeX\else
  % xetex/luatex font selection
\fi
% Use upquote if available, for straight quotes in verbatim environments
\IfFileExists{upquote.sty}{\usepackage{upquote}}{}
\IfFileExists{microtype.sty}{% use microtype if available
  \usepackage[]{microtype}
  \UseMicrotypeSet[protrusion]{basicmath} % disable protrusion for tt fonts
}{}
\makeatletter
\@ifundefined{KOMAClassName}{% if non-KOMA class
  \IfFileExists{parskip.sty}{%
    \usepackage{parskip}
  }{% else
    \setlength{\parindent}{0pt}
    \setlength{\parskip}{6pt plus 2pt minus 1pt}}
}{% if KOMA class
  \KOMAoptions{parskip=half}}
\makeatother
\usepackage{xcolor}
\usepackage{longtable,booktabs,array}
\usepackage{calc} % for calculating minipage widths
% Correct order of tables after \paragraph or \subparagraph
\usepackage{etoolbox}
\makeatletter
\patchcmd\longtable{\par}{\if@noskipsec\mbox{}\fi\par}{}{}
\makeatother
% Allow footnotes in longtable head/foot
\IfFileExists{footnotehyper.sty}{\usepackage{footnotehyper}}{\usepackage{footnote}}
\makesavenoteenv{longtable}
\usepackage{graphicx}
\makeatletter
\def\maxwidth{\ifdim\Gin@nat@width>\linewidth\linewidth\else\Gin@nat@width\fi}
\def\maxheight{\ifdim\Gin@nat@height>\textheight\textheight\else\Gin@nat@height\fi}
\makeatother
% Scale images if necessary, so that they will not overflow the page
% margins by default, and it is still possible to overwrite the defaults
% using explicit options in \includegraphics[width, height, ...]{}
\setkeys{Gin}{width=\maxwidth,height=\maxheight,keepaspectratio}
% Set default figure placement to htbp
\makeatletter
\def\fps@figure{htbp}
\makeatother
\setlength{\emergencystretch}{3em} % prevent overfull lines
\providecommand{\tightlist}{%
  \setlength{\itemsep}{0pt}\setlength{\parskip}{0pt}}
\setcounter{secnumdepth}{5}
\usepackage{booktabs}
\ifLuaTeX
  \usepackage{selnolig}  % disable illegal ligatures
\fi
\usepackage[]{natbib}
\bibliographystyle{plainnat}
\IfFileExists{bookmark.sty}{\usepackage{bookmark}}{\usepackage{hyperref}}
\IfFileExists{xurl.sty}{\usepackage{xurl}}{} % add URL line breaks if available
\urlstyle{same}
\hypersetup{
  pdftitle={e-Campsis documentation},
  pdfauthor={Calvagone},
  hidelinks,
  pdfcreator={LaTeX via pandoc}}

\title{e-Campsis documentation}
\author{Calvagone}
\date{2023-12-08}

\begin{document}
\maketitle

{
\setcounter{tocdepth}{1}
\tableofcontents
}
\hypertarget{about}{%
\chapter{About}\label{about}}

\href{https://ecampsis.shinyapps.io/free/}{e-Campsis} is a free web application developed by \href{https://www.calvagone.com/}{Calvagone} that provides an intuitive and user-friendly interface for setting up population PK/PD simulations.
The app is built on the R-package \href{https://calvagone.github.io/}{campsis}, which serves as a powerful frontend for running model-based simulations using \emph{mrgsolve} or \emph{rxode2}.

\hypertarget{e-campsis-versions}{%
\section{e-Campsis versions}\label{e-campsis-versions}}

\hypertarget{e-campsis-free}{%
\subsection{e-Campsis free}\label{e-campsis-free}}

\emph{e-Campsis free} included many functionalities to provide intuitive and user-friendly interface for setting up population PK/PD simulations.

\hypertarget{e-campsis-free-1}{%
\subsection{e-Campsis free+}\label{e-campsis-free-1}}

\emph{e-Campsis free} has certain limitations regarding the simulation size for unregistered users.

If you want to simulate up to 16 arms or scenarios, 100 subjects/arm and 250 observations/arm we invite you to become an authorized user of \emph{e-Campsis free+}.

Please send us the pre-filled email below and you will get an invitation to register as soon as possible: \href{mailto:campsis@calvagone.com}{\nolinkurl{campsis@calvagone.com}}

\hypertarget{e-campsis-pro}{%
\subsection{e-Campsis pro}\label{e-campsis-pro}}

At \textbf{Calvagone} we are currently working on an advanced version of e-Campsis including the following additional functionality:

\begin{itemize}
\tightlist
\item
  No limitation of number of subjects, observations, ODEs, arms or scenarios
\item
  Save all settings of your simulation project within the Shiny environment
\item
  Import external data into plots for visual comparison to simulations
\item
  Extensive library, including models with categorical endpoints
\item
  Sampling of covariates from external databases like e.g.~NHANES
\item
  Run trial replicates, taking into account parameter uncertainty
\item
  Post-processing of simulation results, applying for example NCA or statistical tests
\item
  Efficient generation of forest plots on derived simulation output (e.g.~Cmax, AUC, \href{mailto:effect@time}{\nolinkurl{effect@time}})
\item
  Semi-automatic parameter sensitivity analysis
\end{itemize}

For further information, contact us at the following e-mail address: \href{mailto:campsis@calvagone.com}{\nolinkurl{campsis@calvagone.com}}

\hypertarget{application-interface}{%
\section{Application interface}\label{application-interface}}

The app consists of 4 main sections:

\begin{itemize}
\tightlist
\item
  \textbf{Model}: a powerful model editor to edit your Campsis model online. Try out one of the numerous models available from the library and adapt it to your needs.
\item
  \textbf{Trial design}: an easy-to-use interface to quickly set-up the dosing regimen, observation times and covariates.
\item
  \textbf{Simulation}: a single screen dedicated to the simulation configuration and visualisation of the results. Explore different scenarios of parameter settings quickly and interactively.
\item
  \textbf{Download}: last but not least, download the model, parameters and the whole code of the simulation to reproduce what you see in the app on your computer using the open-source package campsis.
\end{itemize}

\hypertarget{model-tab}{%
\chapter{Model tab}\label{model-tab}}

\hypertarget{model-from-library}{%
\section{Model from library}\label{model-from-library}}

When entering the app, a simple PK model is already loaded by default.

A different PK model can be selected from a large library (``Select PK model''), or a PD model can be connected (``Connect PD model'') to the PK model. In ``Select category'', NONMEM models or TMDD models can be also loaded.

\hypertarget{campsis-model-import}{%
\section{Campsis model import}\label{campsis-model-import}}

An existing Campsis model can be uploaded from this box (including files \emph{model.campsis}, \emph{omega.csv}, \emph{theta.csv} and \emph{sigma.csv}).

\hypertarget{nonmem-model-import-pro-version}{%
\section{NONMEM model import (pro-version)}\label{nonmem-model-import-pro-version}}

In the pro version, an existing NONMEM model can be uploaded from this box (including files \emph{.mod} and \emph{.ext}) and will be automatically translated to Campsis code.

The NONMEM import functionality will be installed, the process can take several minutes. A notification will popup when done.

\hypertarget{model-code}{%
\section{Model code}\label{model-code}}

The model code is shown in the editor window where it can be easily modified. Please note that the code is case sensitive (e.g.~\emph{log}, \emph{exp}, \emph{sqrt} should be used). The power function is \emph{pow(x,d)}, \emph{x} to the power of \emph{d}.

Clicking on the ``Download'' button, Campsis model code will be downloaded as a ZIP folder, including \emph{model.campsis}, \emph{omega.csv}, \emph{theta.csv} and \emph{sigma.csv}.

\hypertarget{parameters}{%
\section{Parameters}\label{parameters}}

The list of parameters for THETA, OMEGA and SIGMA is given in this box. Their values and labels can be changed. Comments can be added.

The type for OMEGA and SIGMA can be changed: sd, var, covar, cv, cv\%, cor, for standard deviation, variance, covariance, coefficient of variation, coefficient of variation (as \%) or correlation, respectively.

Correlations between omegas can be added by right-clicking on a cell in the OMEGA table. For example, enter ``KA, VC'' as name, 1 and 2 in index and index2, and add the correlation value.

Clicking on ``Get parameter names from code'', the code will be scanned for the \#THETA, \#OMEGA and \#SIGMA and the names will be extracted and added to the table.

Tables can be edited.

\hypertarget{trial-design}{%
\chapter{Trial design}\label{trial-design}}

\hypertarget{trial-design-1}{%
\section{Trial design}\label{trial-design-1}}

Four (free version) to eight (pro version) study arms can be configured.

For each arm tab, the following information can be entered:

\begin{itemize}
\tightlist
\item
  Number of subjects
\item
  Arm label
\item
  Administration type (bolus or infusion)
\item
  If infusion is selected, you can choose whether the infusion is in the Model or in the Dataset; if the latter, the infusion duration can be entered
\item
  Dose amount
\item
  Compartment (in which compartment the dose should be assigned)
\item
  Dosing interval
\item
  Add. doses (number of additional doses)
\item
  Observations (observation time), to be written in R format, e.g.~\emph{seq(0,24,by=1)} or \emph{c(seq(0, 5), seq(0, 5)+168, seq(0,5)+336, seq(0,504,6))}. Enable the ``as-time-after-dose'' box, if you want to replicate the observation schedule after each dose.
\item
  Covariates; e.g.~\emph{BW=70}, \emph{DOSE=1\textbar BW=70}, \emph{WT=NormalDistribution(mean=70, sd=10)}
\item
  Dose adaptation formula (useful if the dose has to be adapted to the body weight); e.g.~DOSE*BW.
\end{itemize}

\hypertarget{summary}{%
\section{Summary}\label{summary}}

A summary of your trial design in shown in this box, where you can quickly visualise the characteristics of your arms.

\hypertarget{custom-dataset}{%
\section{Custom dataset}\label{custom-dataset}}

The simulation dataset (arms) can be further edited by clicking ``Edit dataset'' button.

\href{https://calvagone.github.io/campsis.doc/articles/v01_dataset.html}{See the Campsis help}

\hypertarget{simulation}{%
\chapter{Simulation}\label{simulation}}

Once your trial design is configured, go to the Simulation Tab and the simulation is instantaneously executed.

\hypertarget{scenarios}{%
\section{Scenarios}\label{scenarios}}

Make several scenarios you want to compare. For each scenario, parameter values can be changed.

\hypertarget{simulation-settings}{%
\section{Simulation settings}\label{simulation-settings}}

\begin{itemize}
\tightlist
\item
  IIV/RUV: Should the inter-individual and residual variability be taken into account in the simulations? Check IIV or RUV boxes accordingly.
\item
  Seed: a seed number can be used.
\item
  Select output(s): select one or several outputs you would like to look at.
\item
  Select engine: choose one of the two simulation packages rxode2 or mrgsolve.
\item
  Execution/Manual: check the box to make any changes without updating the plot and, when all is configured, click the ``play'' button \(\vartriangleright\)
\end{itemize}

\hypertarget{plot-settings}{%
\section{Plot settings}\label{plot-settings}}

Click ``+'' to pull the tab down.

\begin{itemize}
\item
  Three plot options can be chosen:

  \begin{itemize}
  \tightlist
  \item
    spaghetti plot: overlay of the individual profiles of the selected output(s) versus time
  \item
    shaded plot: median of the simulated output(s) versus time with 5th and 95th percentiles of the simulations
  \item
    scatter plot: relationship between two selected outputs
  \end{itemize}
\item
  Colour-group by: profiles will have different colors by ARM or SCENARIO
\item
  Stratify-group by: split the plots by ARM or SCENARIO
\item
  X-axis or Y-axis in log: select to show the X- or Y-axis on log scale
\item
  Interactive plot: when checked, more options on plots are available (from Plotly)
\item
  Plot height: adjust the height of the figure
\item
  More annotation options: allows to customize the plot

  \begin{itemize}
  \tightlist
  \item
    Plot title
  \item
    X-axis label, limits, breaks
  \item
    Y-axis label, limits, breaks
  \item
    Footnote
  \item
    Horizontal/Vertical line(s): add one or several horizontal or vertical line(s) to the plot, and select colours and type
  \item
    Facet scales: scales for facet can be fixed, free, or free in one dimension
  \item
    Facet nrow: number of facets per row
  \item
    Facet scaled: include or not the facet variable name
  \end{itemize}
\item
  Custom plot (pro version): code can be edited to directly customize the plot, then check ``enable custom plot'' to update the plot after editing the code. Click ``Generate code from GUI'' to update the code from the plot.
\end{itemize}

\hypertarget{post-processing}{%
\chapter{Post-processing}\label{post-processing}}

Perform post-processing calculations, for example apply non-compartmental methods to derive key PK parameters.

The steps are:

\begin{itemize}
\tightlist
\item
  add the metrics you are interested in (e.g., AUC, Cmax and tmax),
\item
  if relevant, stratify by SCENARIO, ARM and/or PERIOD
\item
  click refresh grid
\item
  click the button `Apply to all panels'
\item
  click on the Calculate button !
\end{itemize}

Description of the options:

\begin{itemize}
\tightlist
\item
  Periods: define the periods to select the time range included for calculation of the metrics
\item
  New metric: select the metric and the output to which the metric should be calculated on, define the time range or select one of the periods defined. A label can be defined. If period is selected, PERIOD has to be selected as a stratification factor
\item
  Available metrics: the metrics created are shown in this box
\item
  Edit zone: click on a metric and dragged it to this zone to edit it
\item
  Drop zone: if a metric had to be remove from one of the stratification, click-and-drop to this zone
\item
  Refresh grid: to reset boxes (on the right)
\end{itemize}

\hypertarget{download}{%
\chapter{Download}\label{download}}

Click the Download button to download the full Campsis script. This script can be used locally on your laptop using the R-package \textbf{Campsis}, available on CRAN. Or the script can be uploaded on e-Campsis later.

\hypertarget{feedback-and-help}{%
\chapter{Feedback and help}\label{feedback-and-help}}

On the \href{https://calvagone.github.io/}{\textbf{Campsis}} website, you will find extensive information about the open-source Campsis simulation platform. There you will also find a section on e-Campsis, including example use cases.

If you run into problems using \textbf{e-Campsis}, please provide your feedback here: \url{https://github.com/Calvagone/ecampsis.feedback/issues}

\hypertarget{disclaimer}{%
\chapter{Disclaimer}\label{disclaimer}}

\textbf{e-Campsis free} is provided ``as is'', without warranty of any kind, express or implied, including but not limited to the warranties of merchantability, fitness for a particular purpose and noninfringement. In no event shall Calvagone SAS be liable for any claim, damages or other liability, whether in an action of contract, tort or otherwise, arising from, out of or in connection with the software or the use or other dealings with the software. Calvagone reserves the right to discontinue the service at any time.

\end{document}
